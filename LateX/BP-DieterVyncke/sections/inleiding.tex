\label{ch:inleiding}

Vandaag worden sites steeds uitgebreider en bevatten steeds meer functionaliteiten. Content beheer is één van deze functionaliteiten die steeds meer gevraagd wordt door de klant. Een content beheerssysteem noemen we een CMS (Content Management System). Deze functionaliteit geeft de eindegebruiker (klant) de optie om zelf bepaalde stukken inhoud aan te passen op hun eigen site. Het voordeel is dat klanten snel, eenvoudig en zonder tussenkomst van een ontwikkelaar, hun site kunnen beheren. 
\newline\newline
Door de vraag naar deze functionaliteit zijn er de laatste jaren steeds meer systemen die zich focussen op CMS. Echter hebben veel van deze systemen hun focus enigszins verlegd. Daar waar de ontwikkelaar een site bouwt met toegevoegde functionaliteit van content beheer, kan de klant nu zijn site volledig zelf opbouwen. Mensen zonder technische kennis betreffende programmeren kunnen op deze manier zelf hun site in elkaar passen. 
\newline\newline
Doordat de functie van ontwikkelaar verlegd is naar de klant, kan dit struikelblokken opleveren voor de ontwikkelaar  zelf. De structuur en de werking van het systeem kunnen anders zijn dan de vertrouwde frameworks die vertrouwd zijn voor ontwikkelaars. Ontwikkelaars die bepaalde kennis en structuren zijn aangeleerd kunnen deze kennis niet ten volle gebruiken in deze CMS'en: ofwel kan het plafond van het CMS bereikt zijn zonder het gewenste resultaat te bekomen, ofwel voldoet het CMS in het algemeen niet aan de verwachtingen. 
\newline\newline
Er zijn verschillende opties in dit geval. De ontwikkelaar schrijft zelf zijn content beheersysteem en begint op die manier volledig vanaf nul. Of je gaat als ontwikkelaar op zoek naar een goed alternatief die de focus behoudt op de ontwikkelaar. Hierbij wordt vermeden dat iedere ontwikkelaar steeds opnieuw het wiel gaat uitvinden.
\newline\newline
De bedoeling van dit stuk is op zoek te gaan naar een goed alternatief voor de klant gefocuste CMS'en. We richten ons op een PHP framework, Laravel. Er wordt op zoek gegaan naar een CMS gebouwd op Laravel dat belangrijke elementen voor een ontwikkelaar kan invullen. Belangrijke elementen bij het kiezen van een nieuw framework is de leercurve, documentatie, community, performantie, beheersbaarheid, databasestructuur enz. Er wordt een vergelijkende testcasus opgesteld met enerzijds een klant gefocust CMS (Drupal), anderzijds een CMS geschreven op Laravel.
\newline\newline
Kan PHP framework Laravel een CMS aanbieden dat een gunstig antwoord biedt op alle of op het merendeel van de opgesomde belangrijkste elementen voor ontwikkelaars? Op welke punten schiet een klant-gefocust CMS tekort?
